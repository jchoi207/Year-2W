\documentclass[8pt]{article}
\usepackage{graphicx} % Required for inserting images
\usepackage[a4paper, total={6.5in, 10in}]{geometry}
\usepackage{amsmath} % Required for \text command
\usepackage{tcolorbox}


\title{PHY294 Thermal Physics}
\author{Jonathan Choi}
\date{}


\begin{document}

\maketitle


\section{Lecture 1: Monday February 26}
\subsection{Introduction}
\begin{itemize}
    \item Particles exhibit \textit{universal} behaviour regardless of composition. When the number of particles $N$ becomes large (in the thermodynamic limit $\sim 10^{23}$, the velocity follows the Maxwell-Boltzmann distribution: $n(v) \propto v^2 e^{-mv^2/2kT}$ 
    \item Historically, it was difficult to quantify the number of particles within everyday samples. Allegedly, around 1773, Ben Franklin estimated the number of molecules within a teaspoon of olive oil to be in the order of $10^{21}$. This is an example of a macroscopic quantity. 
    \item \textbf{Microscopic:} If we wanted to describe a gas by studying its behaviour microscopically, we would account for the positions and velocities of each $N$ particle. 
    \begin{equation} 
    m \ddot{\vec{r_i}} = \vec{F_{i}}(\vec{r_1}, ... , \vec{r_N}, \dot{\vec{r_1}}, ... , \dot{\vec{r_N}}) 
\end{equation}
    
    \begin{itemize}
        \item This represents $3\times 10^{23}$ 2nd-order coupled differential equations. Not possible due to the limits of computation. Thus, the microscopic description is the most rigorous but useless in practical systems.
    \end{itemize}
    \item \textbf{Macroscopic:} consider how we could describe a system by considering it as a whole. We care about thermodynamic quantities: Entropy $S$, Temperature $T$, Pressure $P$, Number of particles $N$ and Volume $V$. These quantities are much easier to measure. 
    \item Ideal Gas (model system): experiments show that a given volume $V$ with $N$ particles eventually comes to a state known as TD equilibrium, where
    \begin{itemize}
        \item The number density of the particles is uniform, and the pressure $P$ and temperature $T$ are uniform and constant in time, so \textbf{no macroscopic fluxes exist}. 
    \end{itemize}
    \item Energy transitions: 
    \begin{itemize}
        \item Thermal Conductivity: Placing two objects at different temperatures in contact 
        \item Convection: Exchange of heat between different states 
        \item Radiation: Exchange of photons between hot and cold objects
    \end{itemize}
\end{itemize}

\section{Lecture 2: Wednesday February 28}
\subsection{Ideal Classical Gas}
\begin{itemize}
    \item Ideal classical gasses have many particles and can be treated as point-like spheres. 
    \begin{equation}
        PV = NkT
    \end{equation}
    \item The Ideal Gas Law is an example of \textbf{universal behaviour}: any gas sufficiently hot and dilute will obey it.

\end{itemize}

\subsection{Deriving the Ideal Gas Law}
\begin{itemize}
    \item We attempt to calculate the macroscopic pressure gas molecules exert in a rectangular tank. To do so, we make assumptions about their velocities, then we find their momentum and thus the force they exert on the tank (which is pressure)
    \item Assumptions:
    \begin{enumerate}
        \item $N \approx 10^{23}$
        \item Gases will bounce elastically with each other and the container 
        \item The density is uniform (in any volume, the molecular density remains the same)
        \item Velocities are isotropically distributed (as many molecules move in one direction as any other) 
        \item There is some average speed that the molecules exhibit
    \end{enumerate}
    \item It is important to note that $\overline{v}_x = \frac{1}{N} \sum v_{xi} = 0$ by isotropy, there are as many molecules with negative velocities in the x direction as there are moving in the positive direction. This is the same with any direction $\bar{v}_y$ and $\bar{v}_z$  equal zero. This is why we define $\bar{v_x}^2 = \sum v^2_{xi}$ as the average speed (just considering the x direction)
    \item  We may consider the average speed in all 3 directions \[\overline{\vec{v^2}} = \frac{1}{N} \sum ||\vec{v}||^2 = \frac{1}{N} \sum v^2_{xi}+ v^2_{yi}+ v^2_{zi} = 3 \bar{v}^2\]
    \item From our derivation (not done here), we have that $kT = m\bar{v}^2 = \frac{1}{3}m\bar{\vec{v}^2}$, using the above velocity expression. Therefore 
    
    \begin{equation}
        KE = \frac{1}{2} m\overline{\vec{v}^2} = \frac{3}{2}kT
    \end{equation}
    
    \item We also define the root mean squared velocity: 
    \begin{equation}
        v_{rms}={\overline{\vec{v}^2}} = \sqrt{\frac{3kT}{m}}
\end{equation}
\item Note: the average kinetic energy of an ideal gas is $\frac{3}{2}kT$ and is only dependent on temperature. However, the $v_{rms}$, depends on the \textbf{molecular mass} of the gas. So, lighter gasses at the same temperature will, on average, travel faster than heavier ones.
\end{itemize}
\\ 
\section{Lecture 3: Friday March 1}
\subsection{Classical Equipartition Theorem}
\begin{itemize}
    \item Temperature as a measure of average kinetic energy is an example of the classical equipartition theorem. 
    \item At thermodynamic equilibrium, the energy of the system will tend to spread out evenly to its thermally active degrees of freedom. The average energy per degree of freedom of a molecule is 
    \begin{itemize}
        \item Translational: $\frac{1}{2}kT$
        \item Rotational: $\frac{1}{2}kT$
        \item Vibrational: $kT$
    \end{itemize}
    \item We can determine a gas's average internal energy from its temperature and composition. This allows us to calculate its heat capacity. 

\end{itemize}

\begin{enumerate}
    \item Monoatomic (e.g. Ne): the monoatomic gas can translate in all three directions. It can technically rotate, but in doing so, it does not change the spatial density (i.e. mass is not moving). It cannot vibrate. So, with $n =3 $ degrees of freedom, it has internal energy $U = 3 \frac{1}{2}NkT$, with specific heat capacity equal to $C_v = \frac{3}{2}Nk$.  (note we can also find the specific heat capacity per molecule by differentiating $C_v$ w.r.t $N$)
    \item Diatomic (e.g. $O_2$): the diatomic gas can translate in three directions, and it can technically rotate along all three axes, but one of them (roll) does not lead to a change in mass density, so $2$. There is one vibrational mode \textbf{; however, this mode might not necessarily be thermally active if temperatures are too low}. Thus a diatomic has internal energy $U = (3 + 2 + 1 \times 2) \frac{1}{2} NkT = \frac{7}{2} NkT$ with specific heat capacity $C_v = \frac{7}{2}Nk$. 
    \begin{itemize}
        \item Note, we multiply the $1$ dof from the vibrational mode by 2, because with each vibration, there are two energies associated: kinetic (movement) and potential (attraction + repulsion). Each vibration is a harmonic oscillator. 
        \item Note that this is only valid given that the gas's temperature is high enough $kT \gg \hbar \omega$. At lower temperatures, the vibrational modes are "frozen out", meaning that they don't have enough energy to occur. 
    \end{itemize}
    \item For linear molecules: $3$ translational, $2$ rotational (for the same reason as diatomics) and $3N-5$ vibrational degrees of freedom. The energy per molecule is  $\frac{kT}{2}(6n-5)$
    \item For non-linear molecules: $3$ translational, $3$ rotational (any rotation leads to change in density) and $3N-6$ vibrational degrees of freedom. The energy per molecule is $kT(3n-3)$  
    \item For solids in a lattice, atoms cannot translate or rotate; however, they may vibrate in all 3 directions, so $f=6$ since each vibrational mode has kinetic and potential. 

\end{enumerate}



\subsection{Heat Capacity}
\begin{itemize}
    \item Heat capacity is not constant with increasing $T$ until $1000 \ \text{K}$. Below this temperature, the heat capacity increases in discrete steps. At lower temperatures, movements are only translational; however, at higher temperatures, the rotational and then vibrational modes become thermally active. Quantum mechanics only allows molecules to vibrate and rotate if they possess some minimum energy.
\end{itemize}


\
\section{Lecture 4: Monday March 4}
\subsection{Heat Transfer and the First Law of Thermodynamics}
\begin{itemize}
    \item Consider how we might change the energy of a thermodynamic system. Place an object in contact with a thermal reservoir at a different temperature. The initial internal energy will be different from the final, since the final temperature $T_f$ changes: $U_{i} = \frac{3}{2}NkT_i$  and $U_f = \frac{3}{2}NkT_f \rightarrow \Delta U = \frac{3}{2}Nk(T_f - T_i)$. We attribute this heat transfer from the hotter system to colder as $Q = \Delta U$ 
    \item Consider a piston compressing a gas. When the wall moves in, the molecules that collide with the piston speed up, increasing $U$ and $T$. When the piston expands, the molecules slow down and both $U$ and $T$ decrease. The work done on the system by pushing the piston, or the work done by the gas on the surroundings by expanding the piston changes the internal energy of the gas $W = \Delta U$ 
    \item As such, the change in internal energy is equal to the sum of the heat absorbed $Q$ and work done on the system $W$. This is also known as the first law of TD: \begin{equation}
        \Delta U = Q + W
    \end{equation}
    \item An alternate form is $dU = \delta Q - pdV$ . Note the negative term is because a compression $dV < 0$ represents work done on the system, and expansion $dV > 0$, is work done on the surroundings, so work is negative. 
\end{itemize}

\subsection{Quasistatic Processes}
\begin{itemize}
    \item A quasistatic process is one where the system is in TD equilibrium at any instant. There is uniform density and pressure and temperature are constant. \textit{The process proceeds slow enough s.t. the system equilibrates at every step}
    \item We may compute the work done on/by the gas in infinitesimal steps. Consider an initial pressure $P_1$ with a compression in volume i.e. $\Delta V_1 < 0$. The work on the gas in this infinitesimal step is $-P_1 \Delta V_1$. Now, we wait for the pressure to equilibrate, and repeat: $W = -p_1 \Delta V_1 - p_2 \Delta V_2 - ... - p_n \Delta V_n$, as $n \rightarrow \infty$, we integrate  $
        W_{\text{on the gas}} = - \int_{V_i}^{V_{f}} \ p \ dV $
        \item However, we don't know the conditions at which the process occurs, so the pressure is a function of $V$ and $T$ \begin{equation}
            W_{\text{on the gas}} = - \int_{V_i}^{V_{f}} \ p(V,T) \ dV 
        \end{equation}
        \item For an ideal gas, $p = \frac{1}{V}NkT$. Now we can consider the characteristics of various processes: 
\end{itemize}

\subsubsection{Isothermal $T = C$}
\begin{itemize}
    \item Consider the expansion or compression at constant temperature. This could be achieved by placing the reaction vessel in a water bath. The work done in an isothermal process is \[W = - NkT \int \frac{dV}{V} = - NkT \ln (\frac{V_f}{V_i})\]
    \item \textit{Since the temperature is fixed, $U$ must remain constant before and after the process. As a result, a compression, compression ($W>0$) requires heat to flow out of the system. Conversely, expansion $W < 0$, requires heat absorption from the reservoir.} $\Delta U = 0 = W + Q \rightarrow W = - Q$. 
    \item Moreover, $pV = C$
\end{itemize}


\subsubsection{Adiabatic $Q =0$}
\begin{itemize}
    \item An adiabatic process is one that is isolated where heat losses or gains are not possible. Adiabatic processes can be both quasistatic and non-quasistatic. In the case of a non-quasistatic adiabate, consider an isolated gas partitioned, into volumes, one with particles, one without. By removing the barrier rapidly, there is no work done. 
    \item In a quasistatic process:
    \begin{itemize}
        \item $pV = NkT$ by the ideal gas law, $U= \frac{f}{2}NkT$ by the equipartition theorem, and $dU = -pdV$ by the first law of TD. By rearranging and plugging things in, we get that $(\frac{V_1}{V_2})^{\frac{2+f}{f}} = \frac{p_2}{p_1}$, this implies that $V_2 ^{\frac{2+f}{f}} p_2 $ is a constant. 
        \item $\frac{2+f}{f} = \gamma >1$ since $f$ is the degrees of freedom, meaning  that the volume and pressure are inversely proportional. \textbf{The adiabatic P-V curve falls steeper than the isotherm.}
        \item $pV^{1 + 2/f} = c$
    \end{itemize}
\end{itemize}

\section{Lecture 5: Wednesday March 6}
\subsection{Heat Capacity}
\\ The heat needed to increase the temperature of a substance by $1$ K. 
\subsubsection{Constant Pressure}
\begin{itemize}
\item Heating at a constant volume means that no work can be done, so $dU = \delta Q$. Applying the equipartition theorem, the internal energy is $U = \frac{f}{2}NkT$, thus, we can find the heat capacity by differentiating w.r.t T. \begin{equation}
    C_v = \left( \frac{\delta Q}{\Delta V}\right)_{V} =  \left( \frac{\partial Q}{\partial V}\right)_{V} 
\end{equation}
    \item The constant volume specific heat capacity is $C_v = \frac{f}{2} kN$
\end{itemize}

\subsubsection{Constant Volume}
\begin{itemize}
    \item Heating a gas up at constant pressure leads to $Q = \Delta U + p\Delta V$, through algebra, we get that $C_p = \frac{f+2}{f} Nk = \gamma C_v$, where gamma is the adiabate constant. As such $C_p > C_v$. 
\end{itemize}


\section{Lecture 6: Thursday March 7}
\subsection{Discrepancy in Heat Capacity}
\begin{itemize}
    \item Consider a hydrogen molecule with interatomic distance of $0.7 \times 10^{-10} \text{ m} $  and mass of $1 \times 10^{-27} \text{ kg}$. Classically, angular momentum is $L = I \dot{\phi}$, and the rotational energy is $\frac{1}{2} I \dot{\phi}^2 = \frac{L^2}{2I}$. 
    \item Recall that the angular momentum is quantized, so we can consider $L=0$ as no rotation, and $L \geq n\hbar$ as rotation. Therefore, the minimum energy of rotation is $E \approx \frac{1\hbar^2}{I}$, calculating $I \approx mr^2 = 1\times 10^{-48} \text{ kg m}^2$, meaning that the minimum energy to observe rotation is on the order of $1\times 10^{-20} J$. Comparing this to $E \approx kT$ , the minimum temperature is $T= 1000 \text{K}$. Thus a temperature of 1000 kelvin is needed to excite rotations. 
\end{itemize}


\subsection{Introduction To Statistical Mechanics}
\begin{itemize}
    \item In an isolated gas of many particles, there are many ways that the energy $E$ can be distributed among the particles. Each of the ways in which the energy can be distributed is called a \textbf{microstate}, whereas the total energy is the \textbf{macrostate}
    \item The fundamental postulate of statistical mechanics states that \textit{all accessible microstates are equally likely}.
    \begin{itemize}
        \item Accessible microstates are those that have total energy that matches a given macrostate
        \item The multiplicity function is the number of microstates accessible to a given macrostate. The multiplicity function $\Omega$ lets you describe the system.
    \end{itemize}
    \item We can no longer describe the system classically with trajectories. Thus, we use probabilistic distributions, and with $N$ large, probabilities become certainties as probability distributions become sharper ($\delta$ functions) 
\end{itemize}

\subsection{Electronic Paramagnet}
\begin{itemize}
\item The macrostate of a paramagnet is the macroscopic quantity of magnetic moment. 
\item We can model the material as a lattice. Each of the $N$ atoms has two degrees of freedom, spin up and spin down: $s_i = \{ +1, -1\}$ for $i = 1 , ..., N$. It is easy to recognize that there are $2^N$possible microstates (each of the $N$ atoms has 2 independent possibilities because we ignore spin-spin and spin-atom interactions).
\item The energy of a single spin in a magnetic field $\mathbf{B}$ is $U_i = -\mu_o \mathbf{B} s_i$, thus the total energy of the system is $U = - \mu_o \mathbf{B} \sum_{i=1}^{N} s_i = -\mu \mathbf{B} S$. where $S$ is the total spin, which is the sum of the individual spins. 

            \item Take the energy of a single spin in a $\vec{B}$ field to be $U_i = -\mu_o B s_i$ note that $\mu_i = \mu_o s_i$ 
   
\[U = - \mu_o B \sum_{i=1}^{N}s_i = - \mu _o B S'\]

\end{itemize}

\section{Lecture 7: Monday March 11}
\subsection{Fundamental Postulate of Statistical Mechanics}
\begin{itemize}
    \item In an isolated gas of many particles, we can describe each individual state's energy as a microstate. The total energy is considered a macrostate. 
    \item The fundamental postulate states \textbf{all microstates are equally likely.} A microstate is said to be accessible if its total energy matches a given macrostate. 
\end{itemize}

\begin{itemize}
    \item For a closed system $U$ fixed, in TD equilibrium all microstates are equally likely. 
\end{itemize}

\subsection{Electronic Paramagnetic}
\begin{itemize}
    \item Recall the spin system with $N$ possible spins with $s_i = \pm 1$ 
    \begin{itemize}
        \item Microstates = $\{s_1, ..., s_N\}$. Thus, there are $2^N$ possible microstates since each $N$ electron has two options  
        \item Macrostate or the total energy is defined by $U = i\mu_o B \sum_{i=1}^{N} s_i = -\mu_o BS$ , where $S$ is the \textbf{total spin}. Note that $S$ is determined by the sum of all microstates. Moreover, $S$ is determined by $N_{up}$ since $N = N_{up} +N_{down}$ and $S = N_{up} - N_{down}$, which implies that $S = 2N_{up} - N$. So $S$ can be fully expressed i.t.o $N_{up}$ . 
        \item We also see that there are $N+1$ possible values of $N_{up}$, thus $S$ has $N+1$ macrostates.
    \end{itemize}
    \item Macrostates are fixed by $S$. We say that a microstate is accessible from the macrostate w/ given $S$ on $N_{up}$ if it has the same $N_{up}$ 
    \item The multiplicity of the general macrostate is the number of accessible microstates given $N_{up}$ is \[\Omega(N, N_{up}) = \frac{N!}{N_{up}! (N-N_{up})!}\]
\end{itemize}
 \subsection{Einstein Solid}
 \begin{itemize}
     \item Collection of $N$ simple harmonic oscillators that function independently of each other. 
     \item The i-th oscillator has energy $\hbar \omega (q_i+ \frac{1}{2})$ where $q_i = 0,1,2, ..., N $ . We ignore the constant of 1/2 (zero-point energy)
     \begin{itemize}
         \item Microstates $= \{q_1 , ... , q_N \}$ so the number of microstates is countably infinite as we take $N$ on the order of Avogadro's number. 
         \item Macrostates = $ U = \sum_{i=1}^{N}q_i = \hbar w q$. Where $q = \sum q_i$  
     \end{itemize}
     \item To find the multiplicity function of the macrostate, we can assume that $N−1$ partitions and $q$ units of energy or objects represent our system. These objects can be partitioned or spread out across the system. \[\Omega(N, q) = \frac{(q+N-1)!)}{q! (N-1)!}\]
 \end{itemize}
\subsection{Example: Energy Flow}
    \begin{itemize}
        \item Consider two systems $N_A$ with $q_A$ and $N_B$ with $q_B$ such that $q_A + q_B$ (the initial states) equals the final states (after the systems come into thermal equilibrium) $q_A + q_B = q_A' + q_B' = c$.
        \item Thus, the multiplicity function of $A$ is $\Omega(N_A, q_A')$ and the multiplicity function of $B$ is $\Omega (N_B, q-q_A')$. Therefore, for the system of $A$ in contact with $B$, the probability of observing some $(q_A', q-q_A')$ is thus proportional to the product between the individual multiplicities $\Omega(N_A,q_A')$ and $\Omega(N_B, q-q_A')$.
        \item Note, for each of the $\Omega_A$ microstates available to $A$, there are $\Omega_B$ microstates available for $B$, hence the product.
    \end{itemize}

\section{Lecture 8: Wednesday March 13}
\subsection{General Thermal Equilibrium - Statistical Definition of Temperature and Energy}
\begin{itemize}
    \item Consider system 1 with energy $E_1$ and system 2 with energy $E_2$ that are brought together and allowed to reach thermal equilibrium. 
    \item In TD equilibrium, the overall system has energy $E = E_1 + E_2$ assuming the process is adiabatic. Thus, we have $E_1$ and $E_2 = E - E_1$ 
    \item The probability of having a particular $E_1$ is $P(E_1)$ is \[P(\Delta) = \frac{\Omega_1(E_1) \Omega_2(E - E_1)}{{\sum^{E/2}_{\bar{\Delta} = -E/2}} \Omega_1(E_1) \Omega_2(E - E_1)} \]
    \item   We aim to find the $E_1$ for the maximum probability of $P(E_1)$. Thus we differentiate the numerator with respect to $E_1$ \[\frac{\partial }{\partial \Delta} \Omega_1(E_1) \Omega_2(E-E_1) = 0\]
    \item Differentiation eventually leads to \[\frac{\partial }{\partial E} k\ln \Omega_1 (E_1) = \frac{\partial }{\partial E} k\ln \Omega_2 (E_2) \implies \frac{1}{T_1} = \frac{1}{T_2}\] 
    \item Each term is a property of a single system, so when they are brought into contact and reach TD equilibrium, their properties, i.e. temperatures must be equal.
    \item We define, entropy as \[S(E, N,V)   = k \ln \Omega (E,N,V)\] 
    \item The inverse temperature of a system is defined as \[\frac{1}{T(E,N,V)} = (\frac{\partial S}{\partial E})_{N,V}\]
    \item \textbf{Some Important Notes:} A closed system evolves s.t. its entropy tends to a maximum value. Since each macrostate must have a multiplicity of at least 1 $\Omega \geq 1$, we must also have $S \geq 0$ 
    \begin{itemize}
        \item In the case of the electronic paramagnetic, the states $N_{up} = N$ only have one microstate, so $\Omega = 1$, $S =0$, thus this system is considered to be very ordered 
        \item If the system is defined by the macrostate  $N_{up} = \frac{N}{2}$ , then $S \gg 1$, so this system is disordered. In this case, \textit{disordered}, refers to the higher number of microstates that are accessible. 
        \item We can also see entropy as the inverse of how much information we have about a system. When $N_{up} = N$, we known exactly what microstate the system is in, but in the case of $N_{up} = \frac{N}{2}$, we have little information about the microstate the system is in. 
        \item Finally notice that $T\geq 0$ because $\frac{\partial S}{\partial E} \geq 0$, so if you add energy into the system, there are more ways to distribute the energy, so entropy increases
            \item Multiplicity is multiplicable $\Omega_{1+2} = \Omega_1 \Omega_2$ since the multiplicity functions are logarithms
            \item Entropy is additive $S_{1+2} = S_1 + S_2$

    \end{itemize}
\end{itemize}

\section{Lecture 9: Thursday March 14}
\subsection{On the Einstein Solid: consideration of the ratio $q/N$}
\begin{itemize}
    \item In the TD limit where $N, V, E \rightarrow \infty$ (all extensive properties), and density and energy density is kept fixed, the results of statistical mechanics become certainties. 
    \item The E.S. was defined with $E = \hbar \omega q$ (zero point energy is ignored) with $\Omega = \frac{(N-1+q)!}{(N-1)!q!}$. Keeping $q/N$ fixed, we consider two constraints $q/N \gg 1$ and $q/N \ll 1$. 
    \item In the \textbf{first limit,} the multiplicity function becomes $\Omega = \frac{(N+q)!}{N!q!}$, recalling that $S =k \ln (\Omega)$, we have that $\ln(\Omega) = \ln (N+q)! - \ln N! - \ln q!$, for large $N$ on the order of $N_a$, we make use of the Stirling approximation: $\ln n! = n \ln n$: \[ \ln(\Omega) \approx (N+1) \ln(N+1) - N \ln N - q \ln q \]
    \item  After some derivation, we get that $\ln(\Omega) = N \ln \frac{qe}{N} \rightarrow \Omega = (\frac{qe}{N})^N$ , thus entropy for the limit of large $q/N \gg 1$ is \begin{equation}
        S = k \ln \left(\frac{qe}{N}\right)^N \rightarrow S = kN \ln \frac{Ee}{\hbar \omega N} 
    \end{equation}
    \item $\frac{1}{T} = \frac{\partial S}{\partial E} = \frac{kN}{E} \implies \frac{E}{N} = kT$. When $q/N \gg 1$, we have that $kT \gg \hbar \omega$, which are the conditions required for equipartition theory.  
    \item In the \textbf{second limit} $q/N \ll 1$, we realize that the multiplicity function is symmetric $\Omega(N, q) = \Omega(q,N)$, so limit 2 is equal to the limit of 1 with $q$ and $N$ switched: $\Omega = (\frac{Ne}{q})^q$, thus entropy is \begin{equation}
        S = k \ln \left(\frac{Ne}{q} \right)^q \rightarrow S = k \frac{E}{\hbar \omega} \ln \frac{N e \hbar \omega}{E}
        \end{equation}
        \item $\frac{1}{T} = \frac{\partial S}{\partial E} = \frac{k}{\hbar \omega} \ln \frac{N\hbar \omega}{E}$, solving for the energy density $\frac{E}{N} = \hbar \omega e^{-\frac{\hbar \omega}{kT}}$. Clearly, equipartition does not hold in the small limit. If the energy of the system is small, the temperature is too low to excite the vibrational modes of the particles, thus the energy per particle drops exponentially.
\end{itemize}


\subsection{Two Einstein Solids}
\begin{itemize}
    \item Consider two Einstein solids with the same number of oscillators: $q_1 + q_2 = q = C$ and both satisfying the limit $\frac{q}{N} \gg 1$ (high temperature regime).  How will the energy distribute among the two systems? $q_1 = q/2 + X$, $q_2 = q/2 - X$. The multiplicity function of both systems is proportional to the probability of observing some state $X$. $P(X) \sim \Omega (q_1, q_2, N, N) = (\frac{e q_1}{N})^N (\frac{e q_2}{N})^N = (\frac{e q/2 + X}{N})^N (\frac{e q/2 - X}{N})^N = (\frac{e^2}{N^2})^N (\frac{q^2}{4} - x^2)^N$. 
    \item Now take the ratio of $\Omega(X)$ to $\Omega(0)$. Through algebra we end up with \begin{equation}
        \frac{P(X)}{P(0)} = \frac{\Omega(X)}{\Omega(0)}= \frac{(\frac{e^2}{N^2})^N (\frac{q^2}{4} - x^2)^N}{(\frac{e^2}{N^2})^N (\frac{q^2}{4})^N} = \left(1- (\frac{x}{q/2})^2\right)^N  
    \end{equation}
    \item \[\ln \frac{P(x)}{P_o} = N \ln \left(1- \left(\frac{x}{q/2} \right) ^2 \right)\]
    \item Where $\frac{x}{q/2}$ is called the relative energy balance. If we consider small energy balance changes, we approximate the above as $-N \left( \frac{x}{q/2} \right)^2$, thus the probability function becomes: \[P(x) = P_o e ^{-N\left( \frac{x}{q/2} \right)^2}\]
    \item Thus, the probability distribution of $X$ (the difference in energies) is a Gaussian that is \textbf{only non-zero for one value}, that value is $x=0$, meaning that the probability of observing some $0$ energy difference between the two systems is a certainty. (The PDF looks like $\perp$, a $\delta$ function)
    \begin{itemize}
        \item We call $\frac{x}{q/2}$ the fractional energy distribution, and is helpful when estimating probabilities. Let us consider a fractional energy imbalance between 1 and 2 of $10^{-5}$ (which is already tiny!). Then $\frac{P(\frac{2x}{q} \sim 10^{-5})}{P(0)} = e^{-10^{23} 10^{-10}} \approx e^{-10^{13}}$. Thus, this essentially says that the probability that the energy imbalance is as small as $10^{-5}$ is essentially 0. 
    \end{itemize}
\end{itemize}


\section{Lecture 10: Monday March 18}
\subsection{Thermodynamic Potential}
\begin{itemize}
    \item Second Law of Thermodynamics: A closed system will evolve towards a TD equilibrium state that maximizes the entropy (subject to imposed constraints) 
    \item Given a closed system with energy $E$, particles $N$ and volume $V$, we define its entropy as  $S(E, N,V) = k \ln \Omega (E, N,V)$. We think of $S$  as a \textbf{TD potential} which determines the approach of TD equilibrium. A similar analogy is how a potential $\mathbf{U}$ defines equilibrium in classical systems $\mathbf{F} = - \nabla \mathbf{U}$. 
    \item $S$ wants to maximize, so the gradient of $S$ as a function of $E, N,V$ (TD coordinates) play a role in the force that drives a system to TD equilibrium. 



\end{itemize}

\subsection{Thermodynamic Potential}
\begin{itemize}
    \item Given a closed system with energy E, particles N and volume V: $S(E, N,V) = k \ln \Omega (E, N,V)$. We can think of $S$ as a \textbf{"TD potential"} which determines the approach of TD equilibrium. In the same way that $\mathbf{F} = - \nabla \mathbf{U}$ defines classical systems. 
    \item $S$ wants to maximize
    \item The gradient of $S$ as a function of $E, N,V$ plays a role of a force that drives the system to TD equilibrium: \[ \frac{1}{T} = \left(\frac{\partial S(E,N,V)}{\partial E}\right)_{N,V}\] 
    \item So we if we have $\frac{1}{T_1}> \frac{1}{T_2}$, then energy will flow until the two are equal 
    \item Note that $\frac{\partial S}{\partial V}$ and $\frac{\partial S}{\partial N}$ are the pressure and chemical potential and are also forced towards equilibrium
\end{itemize}



\subsection{General Properties of Entropy}
\begin{itemize}
    \item From the ES for $\hbar \omega \ll kT $, then, $S \approx kN \ln \frac{Ee}{N\hbar w}$: meaning more energy, more microstates (linear term $N$ beats $\ln(1/N)$). 
    \item Properties to consider:
    \begin{itemize}
        \item The entropy $S$ curve as a function of $E$ is an increasing function, so as $E$ increases, so does $S$ for $T \geq 0$
        \item The second derivative of entropy w.r.t energy, so $\frac{\partial S}{\partial E}$ decreases, and so temperature increases. 
        \item So energy transfer requires $C_v > 0$. All "normal" TD systems have $C_v >0$. Consider two systems with $T_1 > T_2$, as 1 gives away energy, $T_1$ decreases since $C_{v1} >0$, conversely, as energy flows into system 2, $T_2 $ increases since $C_{v2} > 0$ until $T_1 = T_2$. 
        \item Now let's suppose $C_v < 0$, then heat loss leads to an increased temperature, and heat gained leads to a decreased temperature, it is clear that this system has no equilibrium, and one system will end up with all the energy, and one without any. 
    \end{itemize}
    \item Find the multiplicity function, then you get the entropy and can define the system's extensive properties. 
\end{itemize}



\section{Lecture 11: Wednesday March 20}
\subsection{Ideal Gas (Sackur-Tetrode Equation)}
\begin{itemize}
    \item For a given internal energy of a monoatomic ideal gas $U$, how many microstates are there? Recall that the total energy $U$ is equal to the sum of all kinetic energies and thus momenta: \[U = \sum_{i=1}^{N}\frac{\vec{p_i}^2}{2m} \rightarrow \sum_{i=1}^N \vec{p_i}^2  = \sum_{i=1}^N p^2_{xi} + p^2_{yi} + p^2_{zi} = 2mU \] 
    \item Recall the particle in a box problem, the momentum can be expressed as the sum of the squares of the quantum numbers associated with each coordinate: $p_{xi}^2 = \frac{\pi \hbar}{L} n_x$. 
    \item Thus we take the sum of all the quantum numbers which is $3N$ since each particle can move in all three directions: $n_{1x}^2 + n_{1y}^2 + n_{1z}^2 + ... + n_{Nx}^2 + n_{Ny}^2 + n_{Nz}^2 = \frac{2mU}{\frac{\pi \hbar}{\omega}} = R^2$  , which represents a sphere in $\mathbb{R}^{3N-1}$.
    \item Thus, we define the multiplicity function $\Omega(N, U,V) = \text{microstates with energy} \in (U - \frac{\Delta}{2}, U + \frac{\Delta}{2})$. Geometrically this represents two spheres drawn in the $\mathbb{R}^{3N}$ coordinate plane that differ in radius by $\Delta$. $\Omega = A_{3N-1}(R)= \Delta$. Thus, for small enough $\Delta$, the multiplicity function is the area of the sphere times $\Delta$
    \item Corrections: 
    \begin{itemize}
        \item We must partition our space since we only take the positive values ($n_{kx,ky,kz} > 0)$. Thus, we divide by a factor of $2^n$ for each dimension $n$ (Intuition: circle in $\mathbb{R}^2$ gets divided by 4, the sphere in $\mathbb{R}^3$ gets divided by 8)
        \item Gibbs factor: extensive property (if you double all the properties of a system, you would expect entropy to double as well, hence the need for $\frac{1}{N!}$)
    \end{itemize}
\[\Omega(N,U,V) = \frac{2\pi ^{\frac{3}{2}}N}{N! (\frac{3}{2}N-1)! 2^{3N}} \left( \frac{\sqrt{2mU}L}{\pi \hbar} \right)^{3N-1}\]
    \item Now, we find that the entropy $S$ of an ideal gas is (also known as the Sackur Tetrode Equation): 
    \begin{equation}
       S(T, N, V) =  k_B N\left( \ln \left[ \frac{V}{N} \left( \frac{4 \pi m U}{h^2} \right)^{\frac{3}{2}} \right] + \frac{5}{2} \right)
    \end{equation}
    \item Note the following properties of the equation: 
    \begin{itemize}
        \item The entropy $S$ is on the same order as $k_B N$ 
        \item The doubling of all extensive properties $N, U, V$ leads to the doubling of entropy (due to the Gibbs factor)
        \item $S$ increases with $U$ and then reaches a maximum 
        \item If we lower $U$ enough, then the argument of the log becomes less than $1$, and entropy becomes negative. As such, this formula is only valid for sufficiently high energies. 
        \item Similarly, if $\frac{V}{N}$, the volume per particle, becomes small enough (denser gas), the entropy can become negative. 
        \begin{itemize}
            \item Thus, at low temperatures and high densities, the Ideal Gas Law does not apply because entropy becomes negative which is not possible.
        \end{itemize}
    \end{itemize}
    \item Taking the partial derivative to find the temperature: \[\frac{1}{T} = \frac{\partial S}{\partial U} = kN \frac{3}{2} \frac{1}{U} \rightarrow U = \frac{3}{2}NkT\]
    \item Which implies that equipartition is a property of an ideal gas, moreover, we can take $ \frac{P}{T} = \frac{\partial S}{V} = \frac{kN}{V} = \frac{P}{T} \rightarrow pV = NkT$, which is the ideal gas law. 
\end{itemize}


\section{Lecture 12: Thursday March 21}
\subsection{When is a Gas No Longer Ideal?}
\begin{itemize}
    \item We always say that a gas has to be hot and dilute enough. What does "enough" mean? With the Sackur-Tetrode Formula, we can quantify these statements. $\frac{\partial S}{\partial U} = \frac{1}{T} \implies U = \frac{3}{2}kNT$
    \item We introduce the de thermal Broglie wavelength $ \lambda_{th} = \frac{h}{\sqrt{mKT}}$, which is the de Broglie wavelength but with $p = \sqrt{mkT}$; as such, a larger mass and temperature will decrease the thermal wavelength: $\lambda_{th} \propto \frac{1}{\sqrt{mT}}$.
    \item Define the characteristic length between particles as $l = (\frac{V}{N})^{\frac{1}{3}}$. Thus, the ideal gas law holds when $l \gg \lambda_{th}$—where the typical distance between particles is much greater than their thermal de Broglie wavelength, and the quantum effects are not as noticeable. Conversely, when $l \ll \lambda_{th}$—where the typical distance is much smaller than the thermal wavelength, the ideal gas law no longer holds, and the gas is deemed "quantum". 
\begin{equation}
        S \approx kN \ln \left(\frac{l}{\lambda_{th}}\right)^3 
    \end{equation}
    \item We can make a gas "quantum" by decreasing $l$ or increasing $\lambda_{th}$ 
    \begin{itemize}
        \item If we increase the number density of particles, we decrease the $\frac{V}{N}$ term, which decreases the characteristic length $l$
        \item If we lower the temperature, we increase the de Broglie wavelength $\lambda_{th}$
    \end{itemize}
\item These statements make sense as they oppose the behaviour of an ideal gas—low density and high temperatures. To summarize, a gas becomes quantum when $\frac{h}{mkT}=\lambda_{th} \approx l = (\frac{V}{N})^{1/3}$  
\end{itemize}











\end{document}

